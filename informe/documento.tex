\documentclass[a4paper,12pt,twoside]{article}
\usepackage[spanish]{babel}
\usepackage[utf8]{inputenc}
\usepackage{graphicx} %para insertar graficos/imagenes
\usepackage{amsmath} %para escribir matrices
\usepackage{amsfonts} %para poner \mathbb
\usepackage{float} %me deja usar la H de 'here' en los graficos para ponerlos donde yo quiera
\usepackage{anysize} %me permite definir los margenes como quiera
\usepackage{multirow} %para tablas con multicolumna
\usepackage{fancyhdr} % activamos el paquete
\usepackage{dcolumn}
\usepackage{multirow}

\usepackage{fixme}
\fxsetup{
    status=draft,
    author=,
    layout=inline, % also try footnote or pdfnote
}

\newcommand{\grad}{$^\circ$}

\newcommand{\codigoMateria}{66.10}
\newcommand{\nombreMateria}{Circuitos Electronicos II}
\newcommand{\nroTP}{1}
\newcommand{\descripcionTP}{Informe de Laboratorio}
\newcommand{\tituloTP}{Trabajo}
\newcommand{\facultad}{Facultad de Ingeniería}
\newcommand{\universidad}{Universidad de Buenos Aires}
\newcommand{\docentes}{Alberto Bertuccio\\Federico DAngiolo}

\pagestyle{fancy} % seleccionamos un estilo
\fancyhead{}
\fancyfoot{}
\lhead{\nombreMateria \, (\codigoMateria)} % texto izquierda de la cabecera
\rhead{\facultad} % texto centro de la cabecera
\cfoot{\thepage}

\marginsize{2cm}{2cm}{1cm}{1.5cm} %izquierda, derecha, arriba, abajo

\newcommand{\Direcrotio}{./}
\newcommand{\HRule}{\rule{\linewidth}{1mm}}


\newenvironment{items}{
\begin{itemize}
  \renewcommand{\labelitemi}{$\bullet$}
  \setlength{\itemsep}{3pt}
  \setlength{\parskip}{1pt}
  \setlength{\parsep}{1pt}
}{
\end{itemize}}



\begin{document}



\begin{titlepage}

\thispagestyle{empty}

\begin{center}

\includegraphics[scale=0.15]{img/fiuba}\\[0.1cm]
\textsc{\universidad}\\[0.2cm]
\large{\textsc{\facultad}}\\[0.2cm]

\end{center}

\vfill

\begin{center}
\underline{\Large{\nombreMateria\, (\codigoMateria)}}
\end{center}

\vfill
\begin{center}

\end{center}
\vfill

\begin{center}
\Huge{\textsc{ \tituloTP }}\\[.5cm]
	\begin{figure}[H]
		\centering
		%\inclugraphics[width=.5\textwidth]{bessel}
	\end{figure}\HRule \\[0.1cm]
\Huge{\textbf{\descripcionTP}}\\[0.01cm]
\HRule\\[0.3cm]
\end{center}

\vfill



\begin{tabbing}
	FECHA: \today\\
\\
	INTEGRANTES:\hspace{-1cm}\=\+\hspace{1cm}\=\hspace{6cm}\=\\
% 		Gomez, Cristian	\>\>- \#89968\\
%			\>\footnotesize{$<$crisgvenezia@gmail.com$>$}\\
		Pollitzer, Ivan Gustavo	\>\>- \#22922\\
			\>\footnotesize{$<$igpollitzer@gmail.com$>$}\\
		Carballeda, Ignacio	\>\>- \#91646\\
			\>\footnotesize{$<$carballeda.ignacio@gmail.com$>$}\\
		Marques Rojo, Rui Alejandro\>\>- \#85748\\
			\>\footnotesize{$<$rui.rojo@gmail.com$>$}\\			

\end{tabbing}

\begin{flushleft} \large
\emph{Docentes:}\\[.2cm]
\end{flushleft}
\begin{tabbing}
\docentes\\[.5cm]
\end{tabbing}

\vfill

\hrule
\vspace{0.2cm}

\noindent\small{\codigoMateria\, --- \nombreMateria \hfill \facultad}

\end{titlepage}


\newpage
\vfill
\tableofcontents
\vfill

\newpage

\section{Desarrollo}

La arquitectura de los amplificadores esta compuesta generalmente de tres etapas principales, una primer etapa donde entra la señal, se realimenta negativamente la señal de salida y ademas cuenta con un factor de ganancia pequeño; la segunda etapa es de amplificación de tensión y por último la etapa de salida que corresponde a un buffer de corriente.

A partir de la lectura del libro de Douglas-Self sobre el diceño del amplificador clase G pudimos identificar las partes principales que componen el circuito. Como se observa en el siguiente diagrama en bloques.

%\begin{figure}[H]
%	\centering
%	\includegraphics[scale=0.3]{img/bloques1}
%	\caption{Diagrama en bloques del amplificador clase G}
%	\label{fig.1}
%\end{figure}

Proponemos diseñar un circuito que cumpla con las partes principales que se pudieron identificar como así también las especificaciones que decidimos y esperamos llegar a un circuito semejante al que sepuede ver a continuación aunque desde un desarrollo propio.

%\begin{figure}[H]
%	\centering
%	\includegraphics[scale=0.75]{img/claseg}
%	\caption{Diagrama en bloques del amplificador clase G}
%	\label{fig.1}
%\end{figure}








\end{document}
